


The visual analytics for eXplainable AI landscape 
comprehends diverse approaches, each addressing specific aspects of the interpretability challenge through distinct design philosophies and technical implementations. 
To understand how the proposed system relates to existing vXAI tools, this section presents a comparison across multiple dimensions that capture both technical capabilities and user-centered design considerations. 
This section positions the thesis system within the broader context of visual XAI research by comparing it against the five representative tools surveyed in Section~\ref{sec:Visualizations for XAI}: explAIner \cite{8807299}, XAutoML \cite{Z_ller_2023}, SUBPLEX \cite{9861728}, FIPER \cite{cappuccio2024fipervisualbasedexplanationcombining}, and DeforestVis \cite{Chatzimparmpas2023DeforestVisBA}.

The comparison proceeds along key dimensions that capture essential characteristics of visual XAI systems: the visualization architecture and design approach employed, the interaction mechanisms and user workflows supported, and the coordination strategies between multiple views. This multidimensional analysis enables identification of both complementary strengths across different tools and the specific contributions of the thesis system within the synthetic neighborhood and surrogate model paradigm.

The architectural foundation, interaction mechanisms, and coordination strategies of visual XAI systems determine their usability, analytical capabilities, and effectiveness in supporting explanation workflows. These design dimensions shape how users access explanation components, navigate between different analytical perspectives, and build understanding through active exploration. Understanding the integrated design patterns across these dimensions allows for the understanding of the tradeoffs between flexibility, cognitive load management, and analytical depth in the various explainability tools.

explAIner \cite{8807299} employs a toolbox-based architecture where explainers are organized by abstraction level, from high-level model understanding to detailed component analysis, reflecting a progressive disclosure philosophy. The vertically structured interface guides users through increasingly detailed analysis, beginning with high-abstraction overviews before delving into technical details. Interaction mechanisms support progressive exploration where users drill down into specific components through node selection in the TensorBoard graph visualization, with explainer selection triggering the display of results inside the overlay cards. Hover interactions reveal detailed information about specific explanation elements, while the tracking system enables nonlinear workflows by allowing users to mark and save findings. The framework implements coordination primarily through contextual overlays positioned near selected model components, maintaining spatial relationships between components and their explanations without establishing explicit visual connections between different explanation modalities. Multi-model explainers enable comparative analysis through side-by-side examination of different configurations. However, the coordination flow is unidirectional from model graph to explanation displays without reverse feedback. The thesis system adopts a fundamentally different approach centered on the coordination of multiple views rather than a hierarchical toolbox. The parallel presentation architecture simultaneously displays spatial visualization of the synthetic neighborhood and symbolic visualization of the surrogate model structure, reflecting the principle that these complementary representations provide insights users need to access concurrently rather than sequentially. The interaction mechanism allows for bidirectional exploration where users can initiate analysis from either the scatter plot or tree visualization, with clicks in the spatial neighborhood highlighting corresponding decision paths in the tree and vice versa. The bidirectional coordination operates symmetrically, allowing users to verify rule quality by examining spatial distributions of instances satisfying specific predicates, or verify neighborhood quality by examining tree paths of instances in particular spatial regions. 

XAutoML \cite{Z_ller_2023} integrates visualization within JupyterLab through four primary views that segment the AutoML analysis workflow: optimization overview, candidate inspection, search space exploration, and ensemble analysis. The modular architecture emphasizes workflow integration rather than standalone visualization, enabling navigation between different analytical aspects while maintaining context through linked interactions. Hierarchical information presentation through collapsible sections and tabbed interfaces manages the complexity of comprehensive AutoML information. Users interact with the system by selecting candidate models in the optimization overview scatter plot, triggering display of detailed performance metrics, surrogate tree visualizations, and feature importance analyses in the candidate inspection view. Interactive sliders control surrogate model complexity, enabling exploration of interpretability-fidelity tradeoffs, while explicit export buttons facilitate iterative refinement by transferring analysis results back to Jupyter notebooks. Hover interactions provide contextual information following progressive disclosure principles. Coordination implements a master-detail pattern where overview selections control detail view content, enabling model comparison across different ranking criteria with access to comprehensive information for specific candidates. However, similar to explAIner, coordination flows primarily unidirectionally from overview to detail without reverse coordination from detailed views back to overview visualizations. The thesis system shares XAutoML's recognition of modular view organization value and scatter plot selection as an interaction entry point, but applies these principles differently. While XAutoML segments views by analytical task within the AutoML workflow, the thesis system segments views by representational modality (spatial versus symbolic). XAutoML's scatter plot represents candidate models positioned by performance metrics, while the thesis system's scatter plot represents synthetic instances positioned by dimensionality reduction. Both recognize that spatial selection provides intuitive access to detailed information, but the thesis system's addition of tree-to-scatter coordination creates bidirectional workflows absent in XAutoML's unidirectional flow. This difference reflects distinct analytical priorities: XAutoML prioritizes efficient navigation between different candidates and their details, while the thesis system prioritizes confirmation of relationships between spatial distributions and decision logic within a single explanation through symmetric bidirectional coordination.

% FIGURE PROPOSAL: fig:comparison_design_patterns
% Type: Comparison diagram showing integrated design patterns
% Layout: 2x3 grid showing architectural structure, key interactions, and coordination for each tool
% Panel 1: explAIner - vertical toolbox + progressive drilling + contextual overlays
% Panel 2: XAutoML - segmented views + overview-detail navigation + master-detail coordination
% Panel 3: SUBPLEX - five-view layout + iterative refinement + bidirectional coordination
% Panel 4: FIPER - horizontal integration + filtering interactions + within-view coordination
% Panel 5: DeforestVis - five-view workflow + threshold adjustment + multi-directional coordination
% Panel 6: This Work - coordinated dual views + bidirectional exploration + spatial-symbolic linking
% Visual elements: Use consistent visual language showing architecture (boxes), interactions (arrows), coordination (connecting lines)
% Annotations: Label key design elements and interaction flows
% Caption: "Integrated design pattern comparison across visual XAI tools, showing how architecture, interaction mechanisms, and coordination strategies combine to support different analytical workflows"
% Placement: After discussing XAutoML, before SUBPLEX

SUBPLEX \cite{9861728} implements a five-view coordinated interface embedded within Jupyter notebooks: code block for programmatic control, cluster refinement for iterative subpopulation definition, projection view for spatial distribution analysis, subpopulation creation panel for manual group specification, and local explanation detail view for aggregated pattern inspection. This architecture emphasizes flexible subpopulation exploration through multiple coordinated interaction mechanisms, with the projection view employing UMAP to visualize local explanation vector distributions in two-dimensional space. The system provides three distinct interaction modes for subpopulation creation: highlighting instances through code-based filtering, selecting regions through brushing in the projection view, and clicking cluster centroids to select automatically identified groups. These flexible mechanisms acknowledge that domain experts may have specific hypotheses about interesting subgroups beyond automatic clustering. The cluster refinement view enables iterative improvement through feature selection, allowing recomputation of projections and clusters based on feature subsets, with the local explanation detail view providing immediate feedback showing aggregated patterns for selected subpopulations. Coordination operates robustly and bidirectionally between projection and detail views: when users create subpopulations through any selection mechanism, the system immediately updates the explanation detail view; when users refine definitions through feature selection, both projection and detail views update to show revised relationships and patterns. This comprehensive coordination enables iterative exploration where each refinement step provides immediate feedback across all visualization components. The thesis system shares SUBPLEX's architectural emphasis on coordinated views and spatial projection for pattern discovery, with both recognizing that two-dimensional spatial visualizations provide intuitive interfaces for understanding high-dimensional patterns. However, architectural purposes differ fundamentally: SUBPLEX's coordination links spatial clusters of explanation vectors with aggregated explanation patterns across subpopulations, while the thesis system's coordination links spatial positions of synthetic instances with their individual decision paths through the surrogate tree. The thesis system provides similar spatial selection capabilities through scatter plot clicking but differs in refinement mechanisms: SUBPLEX enables iterative refinement of subpopulation definitions themselves through feature selection and re-clustering, while the thesis system enables exploration of fixed neighborhoods through alternative projections and tree layouts. This reflects distinct analytical goals: SUBPLEX supports the discovery of subpopulations with similar explanation characteristics requiring flexible group definition mechanisms, while the thesis system supports understanding of pre-generated neighborhoods requiring flexible representation mechanisms.

FIPER \cite{cappuccio2024fipervisualbasedexplanationcombining} employs a horizontally integrated architecture merging rule visualization with feature distribution displays in a unified interface. The system presents rules as textual predicates on the left while simultaneously showing feature distributions differentiated by type (stacked bar charts for categorical features, box plots for numerical features) on the right, enabling direct comparison between rule predicates and distributional characteristics without view switching. This integrated design reflects FIPER's objective of combining rule-based and feature importance-based explanation modalities. The primary interaction mechanism enables users to filter the feature display to show only attributes mentioned in rule predicates, reducing visual complexity by hiding uninvolved features and supporting focused analysis workflows where users concentrate on understanding decision logic without distraction. Hover interactions reveal detailed distribution information for specific features, including cardinality for categorical variables and statistical summaries for numerical variables, following details-on-demand principles. Coordination operates primarily within this unified view rather than between separate visualizations: the filtering interaction coordinates rule display with feature distribution display through content filtering rather than selection highlighting. The thesis system differs architecturally in several key aspects. Rather than presenting rules as textual predicates, the proposed system visualizes rule structures as interactive decision trees with multiple layout options. Rather than showing feature distributions, the system shows spatial distribution of synthetic neighborhood instances in projected space. Rather than side-by-side integration, the system employs top-bottom coordination with bidirectional interaction mechanisms operating across fundamentally separate visualization types (scatter plot versus tree). While FIPER's filtering interaction operates within a single view to manage complexity, the thesis system's coordination operates across views to establish connections between different representational modalities. Users can examine detailed information about specific elements through hover tooltips in both views while dynamically exploring relationships between spatial positions and decision paths through interactive selection. The thesis system additionally provides interaction mechanisms for exploring alternative representations through dimensionality reduction technique selection and tree layout switching, supporting comparison of different analytical perspectives. These architectural differences stem from different representational priorities: FIPER prioritizes quick rule comprehension through distributional context, while the thesis system prioritizes neighborhood quality assessment and spatial-symbolic confirmation through coordinated exploration, with coordination challenges and solutions differing significantly between integrated layout design and explicit linking mechanisms.

DeforestVis \cite{Chatzimparmpas2023DeforestVisBA} implements a five-view coordinated interface supporting comprehensive surrogate model analysis workflows: surrogate model selection for complexity-fidelity exploration, behavioral model summarization for feature-level aggregation, rule overriding for manual threshold adjustment, decision comparison for impact analysis, and test set results for validation. This architectural design provides multiple abstraction levels for understanding decision stump ensembles, enabling both top-down analysis starting from overall model behavior and bottom-up investigation beginning with individual stumps. Users interact through multiple sophisticated mechanisms: incrementally adding decision stumps to the ensemble while monitoring fidelity metrics to explore complexity-fidelity tradeoffs through interactive model building, manually adjusting decision thresholds within individual stumps in the rule overriding view, and performing case-by-case analysis through instance selection in the test set results view. The rule overriding interaction provides immediate visual feedback showing both local impacts on specific instances and global impacts on overall model predictions, enabling domain experts to incorporate their knowledge into the surrogate model. The system implements the most complex coordination mechanisms among comparison tools, linking five distinct views through multiple interaction options. When users modify surrogate model complexity through the model selection view, updates propagate to all other views: behavioral summarization updates to reflect the new stump ensemble, projection view updates to show modified model behavior, and test results update to show validation performance. When users adjust decision thresholds, the system provides immediate feedback across multiple analytical perspectives simultaneously. This multi-directional coordination enables observation of ripple effects from modifications across all visualization components. The thesis system shares DeforestVis's commitment to coordinated multi-view architecture but organizes views around different analytical priorities. DeforestVis segments views by analytical function within the surrogate exploration workflow (selection, summarization, refinement, validation), while the thesis system segments views by representational format (spatial projection versus tree structure). The thesis system implements simpler coordination limited to bidirectional linking between two primary views, but this focused coordination is specifically optimized for the analytical workflow of exploring synthetic neighborhoods. While DeforestVis's complex coordination supports iterative surrogate refinement across multiple analytical dimensions, the thesis system's focused coordination supports confirmation workflows specific to neighborhood quality assessment. The thesis system does not provide comparable refinement capabilities, reflecting different design priorities: DeforestVis emphasizes interactive surrogate customization as a core workflow enabling users to actively improve or adjust surrogate models, while the thesis system emphasizes interactive neighborhood exploration of automatically generated explanations through multiple representational lenses. The two systems serve different user needs and analytical contexts, with coordination complexity differences reflecting different system scopes.

The synthesis across these integrated design approaches reveals three primary architectural patterns in visual XAI systems. The first pattern, exemplified by explAIner and XAutoML, employs segmented view architectures where different analytical tasks or explanation methods occupy separate interface components accessed through navigation mechanisms, prioritizing comprehensive coverage and workflow integration at the cost of requiring users to mentally integrate information across separated views. Interaction follows progressive exploration paradigms where users navigate from overview to detail through sequential drilling operations, assuming users begin with limited knowledge and gradually build understanding through structured exploration. Coordination implements unidirectional flows where selections in overview visualizations control content in detail views but not vice versa, supporting efficient navigation through hierarchical information structures but limiting confirmation workflows that benefit from reverse coordination. The second pattern, exemplified by FIPER, employs integrated display architectures where complementary information types are presented simultaneously within unified visual structures, prioritizing immediate comparison and reduced cognitive load at the cost of potentially limited space for detailed information presentation. Interaction focuses on managing cognitive load through filtering and details-on-demand mechanisms. Coordination operates within unified views through content filtering rather than between separate visualizations, working well for tightly integrated information types but not scaling to fundamentally different visualization modalities. The third pattern, exemplified by SUBPLEX, DeforestVis, and the thesis system, employs coordinated view architectures where multiple linked visualizations present different perspectives on related information with synchronized interaction mechanisms, balancing the coverage benefits of segmented views with the comparison benefits of integrated displays through explicit coordination mechanisms that maintain visual connections across separated components. Interaction emphasizes iterative refinement where users modify, filter, or reorganize information to test hypotheses and discover patterns, assuming users have working hypotheses they want to investigate through active manipulation. Coordination implements bidirectional or multi-directional patterns where interactions in any view propagate updates to other views, supporting iterative exploration and confirmation workflows while requiring careful design to avoid confusing feedback loops or cognitive overload from excessive visual updating.

The thesis system's contribution within this design landscape lies in its coordinated architecture specifically optimized for bidirectional spatial-symbolic linking in synthetic neighborhood exploration contexts. The parallel presentation architecture enables fluid transitions between spatial exploration of neighborhood quality and symbolic exploration of extracted rules, supporting confirmation workflows where users verify explanation quality by examining correspondence between synthetic instance distributions and decision boundaries. The bidirectional coordination mechanism directly addresses a fundamental challenge in neighborhood-based local explanations: assessing whether generated neighborhoods adequately capture local decision boundary characteristics. By enabling users to select spatial regions and immediately observe corresponding decision paths, or select decision paths and immediately observe corresponding spatial distributions, the system transforms neighborhood quality assessment from an abstract statistical concern into a concrete visual confirmation task. The provision of multiple tree layout alternatives within this coordinated framework further enhances effectiveness by allowing users to choose representational formats matching their cognitive preferences while maintaining consistent coordination behavior across layout alternatives. The ability to initiate exploration from either spatial or symbolic entry points accommodates different user mental models and analytical approaches: users who think spatially can begin by examining neighborhood distribution patterns and then investigate decision logic for instances in interesting regions, while users who think symbolically can begin by examining tree structures and then investigate spatial distributions of instances satisfying specific predicates. This flexibility, combined with dimensionality reduction alternatives, provides rich support for diverse exploration strategies within the focused domain of synthetic neighborhood analysis, distinguishing the system from both hierarchical progressive disclosure approaches and single-view integrated approaches while sharing the coordinated view paradigm with tools like SUBPLEX and DeforestVis but applying it to the specific analytical challenges of neighborhood-based local explanations.






Once compared the proposed system with the surveyed visualizations systems, this section compares the thesis's proposed tree layouts against previously analyzed decision tree visualization approaches surveyed in Section \ref{sec:Visualizations for XAI}, examining how the system integrates proven design patterns and contributes to synthetic neighborhood analysis tailored solutions. The comparison proceeds via five complementary dimensions: visual representation strategies, layout algorithms and spatial organization, information encoding and node design, interaction paradigms and progressive disclosure, and integration with complementary visualization components.

\subsection{Visual Representation Strategies}

The choice of the representation strategy directly influences interpretability, as different visual approaches emphasize different aspects of tree structure and decision logic. This dimension examines how various approaches balance structural clarity, space efficiency, and cognitive accessibility.

The literature reveals substantial diversity in representation approaches. Streeb et al. \cite{Streeb2021TaskBasedVI} demonstrate through systematic analysis of 152 publications that node-link diagrams represent the dominant paradigm, appearing in 110 publications. This prevalence stems from their intuitive mapping of hierarchical relationships through explicit parent-child connections. Traditional implementations employ circular nodes connected by edges, with vertical positioning indicating depth and horizontal spreading accommodating branching structure \cite{schulz2011treevis, elzen2011baobabview}.

Alternative representation strategies address specific limitations of traditional node-link diagrams. Radial layouts, documented extensively by Schulz \cite{schulz2011treevis} and illustrated in Figure \ref{fig:radial_evolution}, transform vertical hierarchies into circular arrangements where tree depth maps to radial distance from center. This approach optimizes space utilization for balanced trees but introduces challenges for path tracing in deep structures. Indentation diagrams provide text-based alternatives where parent-child relationships emerge through spatial indentation, though Elzen et al. \cite{elzen2011baobabview} note these suffer from poor structural overview for large trees.

Space-filling approaches, including icicle plots and treemaps maximize information density by partitioning available space proportionally to tree structure. Ming et al. \cite{ming2019rulematrix} introduce matrix-based representations through RuleMatrix, where decision trees transform into tabular rule lists with rows representing decision paths and columns representing features. This representation strategy abandons explicit hierarchical visualization in favor of standardized rule comparison, as demonstrated in Figure \ref{fig:rulematrix_pipeline}. Three-dimensional approaches, explored by Mrva et al. \cite{mrva2019decision}, employ circular plane layouts with 3D bar charts for medical data analysis, shown in Figure \ref{fig:3d_medical_trees}, though spatial depth perception challenges limit widespread adoption.

The thesis system employs three variations of the node-link representation strategy, each adapting the traditional hierarchical visualization to emphasize different analytical priorities. All three layouts maintain explicit parent-child relationships through connected nodes and edges, preserving the intuitive hierarchical structure that dominates the literature.
The \textbf{Tree Layout} visualization implements a traditional top-down node-link diagram with circular nodes, following the conventional Reingold-Tilford positioning algorithm \cite{1702828}. This layout prioritizes immediate recognizability and leverages established cognitive patterns for tree interpretation, providing users with the familiar hierarchical structure employed by the majority of surveyed publications.
The \textbf{Rule and Counterfactual Rules Centered} visualization enanches the node-link paradigm through two distinctive features: rectangular node geometry and depth-aligned positioning where all nodes at equivalent tree depths occupy the same horizontal level. This creates a more structured, grid-like appearance while maintaining explicit hierarchical connections through edges. Crucially, this layout guarantees bottom positioning for the explained instance path, establishing consistent spatial anchoring that facilitates rapid identification of the factual rule and enables direct comparison with counterfactual alternatives positioned in parallel horizontal rows above.
The \textbf{Rule Centered} visualization implements a hybrid visual strategy within the node-link framework. The explained instance path receives enhanced prominence through rectangular node representation and horizontal linear positioning. Alternative branches stemming from path nodes maintain traditional hierarchical organization with circular nodes, positioned above and below the central path using standard tree layout algorithms. This design distinguishes explanation-relevant information through shape differentiation (rectangles vs. circles) and spatial prominence (central horizontal path vs. peripheral subtrees) while preserving complete hierarchical structure through explicit node-edge connections.

\subsection{Layout Algorithms and Spatial Organization}

Layout algorithms determine node positioning and space allocation strategies, fundamentally shaping how effectively tree structure communicates to users. Optimal layout algorithms balance competing objectives including edge crossing minimization, uniform space utilization, structural clarity preservation, and scalability to varying tree complexities. This dimension examines how different algorithmic approaches address these objectives and support different analytical workflows.

The Reingold-Tilford algorithm \cite{1702828} represents the established foundation for node-link tree layouts, implementing a two-pass positioning strategy that ensures nodes at equivalent depths occupy the same vertical level while minimizing horizontal spread. The algorithm's prevalence stems from proven ability to create aesthetically pleasing layouts without node overlaps, as documented by Schulz \cite{schulz2011treevis} across diverse tree visualization techniques.

Layer-based organization, identified by Schulz \cite{schulz2011treevis} as essential for preserving topology understanding, appears consistently across surveyed approaches. Wang et al. \cite{wang2022timbertrek} demonstrate sophisticated layer organization through Sunburst diagrams that implement focus-context techniques \cite{readingsInformationVi}, allowing transitions between overview and detail levels. Elzen et al. \cite{elzen2011baobabview} implement weighted edge algorithms that optimize readability through intelligent spacing adjustments based on data flow characteristics, visible in Figure \ref{fig:baobab_partitioning}.

The thesis system's Tree Layout employs the established Reingold-Tilford algorithm with dynamic scaling mechanisms that adapt horizontal and vertical spacing proportionally to total node count and ensures larger trees receive proportionally more space while maintaining consistent visual density. A separation function enforces a minimum gap between adjacent nodes at equivalent depth levels, preventing visual congestion regardless of local branching structure. This implementation prioritizes familiarity and proven effectiveness over algorithmic innovation, acknowledging the Reingold-Tilford algorithm's widespread validation.

The Rule and Counterfactual Rules Centered visualization implements uniform horizontal spacing per depth level, creating depth-aligned columns where all nodes at equivalent depths occupy identical horizontal positions. This strategy departs from the Reingold-Tilford algorithm's optimized spreading, instead prioritizing comparison across branches at equivalent decision depths. Vertical positioning employs sorted placement based on branch divergence points, organizing paths to minimize visual crossing and facilitate counterfactual comparison. This depth-alignment approach shares conceptual similarities with coordinate-based visualizations including parallel coordinates \cite{elzen2011baobabview, 10.1007/978-3-540-74205-0_121}, where consistent axis positioning enables direct feature comparison across instances. However, where parallel coordinates sacrifice hierarchical structure for comparison optimization, the Rule and Counterfactual Rules Centered layout maintains parent-child relationships through explicit edges.

The Rule Centered visualization implements an adaptive spacing mechanism defined through Equation \ref{eq:SpawnTreeSpacing}, where spacing margin between consecutive path nodes depends on off-path subtree size. This approach ensures complex tree regions receive adequate space while compact regions maintain efficient layouts and is a contribution in fixed-spacing approaches, dynamically balancing space efficiency against structural clarity based on actual tree characteristics rather than predetermined spacing rules.

The bottom-anchoring strategy employed in Rule and Counterfactual Rules Centered represents a unique positioning contribution. Where traditional layouts position nodes based purely on hierarchical relationships and TIMBERTREK employs radial focus-context \cite{wang2022timbertrek}, the guaranteed bottom positioning for explained instance paths establishes consistent spatial expectations. Users can immediately locate explanation-relevant information without visual search, reducing cognitive load for repeated explanation examinations. This strategy shares conceptual similarities with focus-context techniques but implements focus through consistent spatial assignment rather than dynamic magnification or repositioning.

Scalability trade-offs emerge clearly across the three layouts. Tree Layout handles deep trees effectively through vertical expansion but struggles with wide trees where extensive horizontal spreading challenges viewport constraints. Rule and Counterfactual Rules Centered manages wide trees efficiently through depth-alignment but faces vertical expansion challenges when many paths diverge from common ancestors. Rule Centered optimizes for explanation path visibility but requires careful subtree collapse management for complex trees with extensive branching. These complementary scalability profiles justify the multi-layout approach, as different tree characteristics favor different spatial organizations.

\subsection{Information Encoding and Node Design}

Information encoding strategies determine how tree components visually represent underlying data characteristics, decision logic, and prediction confidence. Effective encoding balances information density against perceptual clarity, ensuring users can rapidly extract relevant information without overwhelming cognitive processing. This dimension examines visual variables including node size, shape, color, internal content, and edge properties.

Wang et al. \cite{wang2022timbertrek} demonstrate funnel-like node representations where node width corresponds to training sample percentage, enabling rapid identification of important decision points and model robustness assessment. This size encoding strategy provides immediate visual feedback about data flow distribution across tree structure. Elzen et al. \cite{elzen2011baobabview} showcase multi-layered node content integrating class distributions via streamgraphs, attribute values, and split conditions within compact visual representations, as demonstrated in Figure \ref{fig:baobab_interface}. Ming et al. \cite{ming2019rulematrix} demonstrate compact clause representations combined with data distribution previews using histograms for continuous features and bar charts for categorical features.

Color coding strategies emphasize accessibility and consistency. Parr et al. \cite{parr2019dtreeviz} uses an handpicked colorblind-safe palettes with distinct schemes for different numbers of target categories, ensuring accessibility while maintaining visual coherence across complex trees. Elzen et al. \cite{elzen2011baobabview} employ attribute-based node coloring revealing which features most frequently drive splitting decisions. Ming et al. \cite{ming2019rulematrix} use higher opacity for satisfied conditions and gradient coloring for confidence intervals, while Mrva et al. \cite{mrva2019decision} implement transparent rendering for filtered nodes, allowing focus on specific classes while maintaining context.

Edge representation encodes data flow through variable width and color. Elzen et al. \cite{elzen2011baobabview} pioneer color-banded edges with variable width visualizing instance flow volume through decision branches, providing immediate visual feedback about data distribution across tree structure, as shown in Figure \ref{fig:baobab_partitioning}. Parr et al. \cite{parr2019dtreeviz} employ thicker, colored edges for paths involved in specific predictions. Joesquito \cite{joesquito2024decision} demonstrates animated traversal with color transitions showing decision progression in real-time.

Data distribution integration represents an advanced encoding pattern. Parr et al. \cite{parr2019dtreeviz} embed strip plots and scatter plots within decision nodes showing actual data distributions for split conditions, eliminating cognitive burden of connecting abstract tree structure with underlying data patterns, as illustrated in Figure \ref{fig:tool_comparison}. Ming et al. \cite{ming2019rulematrix} implements conditional distribution visualization showing feature distributions given previous rules are not satisfied, providing crucial context for understanding rule interactions.5

The thesis system employs consistent class-based color schemes across all three layouts, maintaining spatial-symbolic coherence between tree visualizations and the neighborhood scatter plot. The color palette, up to 10 target classes, follows the colorblind-safe schemes proposed by Parr et al. \cite{parr2019dtreeviz}, with handpicked colors adapted for different numbers of target categories to ensure accessibility and visual distinctiveness. This coordination prioritizes cross-visualization consistency over within-tree sophistication, enabling users to trace instances between spatial and symbolic representations through color matching. Split nodes employ neutral gray coloring emphasizing decision-making roles without competing with class-specific colors, while leaf nodes receive class-consistent coloring matching scatter plot encoding, maintaining the same colorblind-safe palette throughout the interface.

Variable-width edges appear consistently across all three thesis system layouts, encoding instance flow proportionally through stroke width. This encoding matches the pattern established by BaobabView \cite{elzen2011baobabview}, though the thesis system employs this encoding specifically for neighborhood-based explanations rather than general model analysis. Thicker edges indicate paths with more synthetic instances, providing immediate visual feedback about neighborhood distribution across decision structure. Edge color additionally encodes split outcomes, with negative ramifications rendered in red and positive outcomes in green, enabling rapid visual parsing of logical conditions.

Text integration in rectangular nodes represents a more aggressive strategy than dtreeviz's embedded distributions \cite{parr2019dtreeviz}. Where dtreeviz integrates statistical visualizations within nodes while maintaining compact circular geometry, the thesis system's Rule and Counterfactual Rules Centered and Rule Centered path nodes embed textual split conditions and class predictions directly, sacrificing space efficiency for reduced interaction requirements. Split conditions appear as readable text within node bounds rather than requiring hover interactions, prioritizing rapid condition comprehension for neighborhood analysis workflows.

Tooltip-based progressive disclosure provides comprehensive contextual details through details-on-demand interaction. Split node tooltips reveal feature names, threshold values, impurity measures, sample counts, and class distribution statistics. Leaf node tooltips display class predictions, confidence measures, and sample counts. This layered information architecture allows users to access varying detail levels based on analytical requirements. However, this strategy introduces a trade-off between reduced cognitive load through initial information hiding and increased interaction burden through required hover actions. Alternative approaches like BaobabView's integrated streamgraphs \cite{elzen2011baobabview} and dtreeviz's embedded distributions \cite{parr2019dtreeviz} make statistical information immediately visible without interaction, though at the cost of increased visual complexity.

The thesis system deliberately avoids sophisticated within-node statistical visualizations present in BaobabView and dtreeviz. This design choice reflects differing system objectives: BaobabView and dtreeviz focus on tree-centric analysis where feature distributions and class separations constitute primary analytical targets, while the thesis system prioritizes neighborhood quality assessment and rule exploration where spatial distribution (visible in scatter plot) provides feature context. The system accepts reduced within-node information density in exchange for enhanced cross-visualization coordination and simpler node designs that prioritize rule readability.

\subsection{Interaction Paradigms and Progressive Disclosure}

Interaction mechanisms determine how users navigate complexity, explore alternative scenarios, and confirm understanding through active engagement. Effective interaction paradigms balance immediate information access against cognitive overload, supporting diverse analytical workflows without overwhelming users with options or requiring extensive training. This dimension examines fundamental interaction patterns including details-on-demand, dynamic filtering, progressive disclosure, and coordinated views.

Details-on-demand represents the foundational interaction pattern appearing across virtually all surveyed systems. Hover tooltips provide contextual information without cluttering primary visualization, as demonstrated in BaobabView \cite{elzen2011baobabview}, TIMBERTREK \cite{wang2022timbertrek}, dtreeviz \cite{parr2019dtreeviz}, and RuleMatrix \cite{ming2019rulematrix}. The details-on-demand paradigm \cite{readingsInformationVi} allows users to selectively access detailed information for elements of interest without overwhelming the primary representation.

Dynamic filtering and real-time manipulation enable users to explore alternative model configurations or focus on specific aspects. RuleMatrix \cite{ming2019rulematrix} implements rule filtering by support and confidence thresholds, allowing users to hide low-quality rules and focus on robust patterns. Mrva et al. \cite{mrva2019decision} enable class-based filtering with transparency adjustments, letting users emphasize specific prediction categories. BaobabView \cite{elzen2011baobabview} demonstrates real-time threshold modification with immediate visual feedback, as shown in Figure \ref{fig:baobab_algorithmic_support}, enabling interactive tree construction where users adjust split criteria and observe resulting structure changes.

Progressive disclosure manages complexity through incremental information revelation. RuleMatrix \cite{ming2019rulematrix} implements expandable cells revealing detailed feature distributions on demand, balancing comprehensive information access against initial visual simplicity. Wang et al. \cite{wang2022timbertrek} provide repositionable tree windows enabling comparative analysis across multiple model candidates from Rashomon sets, demonstrated in Figure \ref{fig:timbertrek_system}. This pattern supports exploratory workflows where users progressively reveal complexity as analytical needs evolve rather than confronting complete information immediately.

The thesis system implements hover tooltips providing node-specific context including split conditions, sample statistics, impurity measures, and class distributions. These tooltips maintain consistent information architecture across all three layouts, ensuring users encounter familiar interaction patterns regardless of selected visualization. Click interactions trigger cross-visualization coordination rather than within-tree manipulation, as leaf node clicks highlight all instances satisfying complete logical paths in the scatter plot, while split node clicks highlight instances passing through specific decision points.

The Rule Centered visualization implements collapsible subtree functionality through right-click context menu interactions, as illustrated in Figure \ref{fig:spawnSubtreeInteractions}. Subtrees not involved in explained instance paths can be collapsed to reduce visual complexity, allowing users to focus on explanation-relevant information while maintaining options to explore alternative decision paths when needed. Collapsed subtrees appear as simplified placeholder nodes with special visual marking indicating hidden structure. This progressive disclosure mechanism supports exploratory analysis workflows where users progressively reveal complexity as needed rather than confronting full tree structure immediately.

The absence of progressive disclosure in Tree Layout and Rule and Counterfactual Rules Centered visualizations reflects deliberate design choices based on differing analytical objectives. Tree Layout prioritizes complete structural overview, supporting users who benefit from simultaneous access to entire decision hierarchy. Rule and Counterfactual Rules Centered emphasizes counterfactual comparison through depth-alignment, where complete path visibility facilitates branch comparison. Rule Centered addresses different needs by focusing attention on explained instance paths while maintaining access to contextual branches through progressive disclosure. These complementary interaction models acknowledge that different analytical workflows benefit from different information management strategies.

The thesis system emphasizes tree-to-neighborhood coordination over within-tree manipulation. No threshold modification, tree editing, or split criteria adjustment capabilities appear in any layout. This design choice reflects fundamental differences in system purpose relative to tools like BaobabView \cite{elzen2011baobabview} and RuleMatrix \cite{ming2019rulematrix}. Those systems support interactive model construction and refinement, where users actively shape decision structures through parameter adjustment and algorithmic guidance. The thesis system instead focuses on exploring pre-generated explanations, where synthetic neighborhoods and surrogate models emerge from genetic algorithms and training processes. Users analyze existing explanations rather than constructing new ones, shifting interaction emphasis from model manipulation to explanation confirmation through spatial-symbolic coordination.

Layout switching represents a meta-interaction enabling users to select among three representation alternatives based on cognitive preferences and analytical objectives. This switching mechanism addresses diverse analytical needs complementarily to TIMBERTREK's focus-context \cite{wang2022timbertrek}. Where TIMBERTREK provides multiple simultaneous views of the same tree through repositionable windows, the thesis system enables users to select single layouts optimized for specific tasks. Tree Layout supports general structure comprehension, Rule and Counterfactual Rules Centered facilitates counterfactual comparison, and Rule Centered focuses attention on explanation paths. This approach acknowledges that different users approach explanation interpretation differently, providing flexibility without requiring users to learn multiple tools.

The highlighting mechanism differs from animated traversal approaches demonstrated by Joesquito \cite{joesquito2024decision}. Where animated traversal creates temporal narratives through sequential highlighting with color transitions, the thesis system employs static highlighting with persistent visual feedback. This trade-off sacrifices narrative flow for immediate recognition and reduced animation distraction. Users examining multiple instances in rapid succession benefit from instantaneous highlighting updates rather than waiting for animation sequences to complete. However, animated approaches may support users who benefit from temporal sequencing for understanding logical flow, suggesting potential future enhancement directions.
